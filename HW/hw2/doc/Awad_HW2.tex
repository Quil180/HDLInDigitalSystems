\documentclass[11pt, letterpaper]{article}

% Packages
\usepackage[utf8]{inputenc}
\usepackage{geometry}
\usepackage{graphicx}
\usepackage{listings}
\usepackage{xcolor}
\usepackage{amsmath}
\usepackage{float}
\usepackage{hyperref}
\usepackage{caption}

% Page Layout
\geometry{margin=1in}

% Code Listing Style
\definecolor{codegreen}{rgb}{0,0.6,0}
\definecolor{codegray}{rgb}{0.5,0.5,0.5}
\definecolor{codepurple}{rgb}{0.58,0,0.82}
\definecolor{backcolour}{rgb}{0.95,0.95,0.92}

\lstdefinestyle{verilogstyle}{
    backgroundcolor=\color{backcolour},   
    commentstyle=\color{codegreen},
    keywordstyle=\color{blue},
    numberstyle=\tiny\color{codegray},
    stringstyle=\color{codepurple},
    basicstyle=\ttfamily\footnotesize,
    breakatwhitespace=false,         
    breaklines=true,                 
    captionpos=b,                    
    keepspaces=true,                  
    numbers=left,                    
    numbersep=5pt,                  
    showspaces=false,                
    showstringspaces=false,
    showtabs=false,                  
    tabsize=2,
    language=Verilog
}

\lstset{style=verilogstyle}

% Title Info
\title{HDL Design Homework: SystemC vs SystemVerilog}
\author{Yousef Awad \\ EEL 4783 - HDL in Digital Systems}
\date{\today}

\begin{document}

\maketitle

\section*{Part 1: SystemC}

\subsection*{1.1: Counter Module}
\subsubsection*{a) The completed module skeleton}
\lstinputlisting{../src/counter.cpp}

\subsubsection*{b) Why is SC\_METHOD (not SC\_THREAD) appropriate here?}
\texttt{SC\_METHOD} is ideal because it executes its entire block of code and returns control to the simulator without needing to be suspended. Since the requirements explicitly state not to use \texttt{wait()}, \texttt{SC\_METHOD} is the correct choice over \texttt{SC\_THREAD} (which requires \texttt{wait()} to suspend and resume execution).

\pagebreak
\subsection*{1.2: Testbench Snippet}
\lstinputlisting{../src/main.cpp}

\newpage
\section*{Part 2: SystemVerilog}

\subsection*{2.1: Shift Register RTL}
\lstinputlisting{../src/shift_reg.sv}

\subsection*{2.2: Testbench Logic}
\lstinputlisting{../src/shift_reg_tb.sv}

\subsection*{2.3: Concept Check}
\textbf{a) Why does SystemC use sc\_signal instead of plain C++ variables for ports?}\\
\texttt{sc\_signal} implements the evaluate-update paradigm of hardware simulation, ensuring that updates are deferred until the next simulation delta cycle to prevent race conditions that plain C++ variables would cause.

\vspace{1em}
\noindent
\textbf{b) What critical error occurs if you declare data\_out as wire instead of logic in Problem 2.1?}\\
If you attempt to assign to \texttt{data\_out} procedurally (inside the \texttt{always\_ff} block) while it is declared as a \texttt{wire}, the SystemVerilog compiler will throw an error, because \texttt{wire} types can only be driven by continuous assignments (\texttt{assign}), whereas \texttt{logic} can be driven procedurally.

\end{document}
